%----------------------------------------------------------------------------------------
%	CHAPTER 1
%----------------------------------------------------------------------------------------

%\chapterimage{head2.png} % Chapter heading image


\chapter{Introduction}

Internet usage/data transfers in some of the Pacific Islands are problematically slow ~\cite{4}. Many of these islands do not have the infrastructure or budget for undersea fibre-optic cable connected Internet but rather rely on satellite link Internet ~\cite{3}. Samoa and Tonga, for instance, have only recently had these fibre-optic cables connected to the high-rate Internet but unfortunately, many of the smaller islands still rely on satellite connectivity ~\cite{3}. The result is that many Pacific Islands are left with slow, archaic Internet data rates due to the latency and bottleneck bandwidth of the satellite link. This thesis will discuss the problems this creates ~\cite{4}. This work looks at options to help solve these issues and creates the foundation for a performance enhancing proxy (PEP) that primarily aims to counteract the factors creating slow Internet data rates. In general, most PEPs give TCP senders the illusion that packets are traveling over a shorter distance and thus help solve some of the issues created by latency and the bottleneck nature of the satellite link ~\cite{6}.This thesis first discusses a variety of PEP concepts and then introduces our variation on the theme: a \emph{non-connection breaking PEP}. In this chapter, we will briefly look at the problem and motivation for our thesis. 

\section{Problem And Motivation}\label{Motivation}

\emph{The problem:} In a nutshell, TCP originated in terrestrial networks (of comparatively low latency) in all its variant forms (TCP Reno, TCP Cubic etc.) does not work well with narrowband satellite links ~\cite{5}. On satellite links, the latency between the sender and receiver is higher than on purely terrestrial links. It is up to 500 ms round trip time (RTT) or more via geostationary (GEO) satellites, 120 ms or more via medium earth orbit satellites (MEO). The further problem is that the narrowband nature of many satellite links creates a bandwidth bottleneck at which congestion can occur ~\cite{4}. This combination of latency and bottleneck is the root of the problem which this thesis will offer more in-depth explanations on later ~\cite{7}~\cite{8}. The symptoms of this problem, at least from an Internet users perspective, are evidenced by excruciatingly slow Internet speeds. For these Internet users connected solely to a satellite link, such as those living in the Pacific Islands mentioned previously, web browsing becomes frustratingly slow and large downloads, or streaming is next to impossible ~\cite{4}. The symptoms of this problem from a network perspective are the underutilisation of the link, excessive queue oscillation at the input to the satellite link, unnecessary retransmissions and false timeouts which all contribute to degraded Internet service for users ~\cite{8}~\cite{9}. (Chapter 2 will discuss these problems in more depth). The essential elements that comprise our problem stem from TCP's mechanism to guarantee reliability and TCP congestion control ~\cite{10}~\cite{11}. \\

The transport layer protocol (TCP) ensures that all data sent for applications between two hosts is delivered with reliability, integrity, in-order, and without duplication . TCP reliability means that the data delivery is reliable and guaranteed to reach the receiver and both receiver and sender are informed of successful deliver. Integrity assures that the data is not corrupted and is received with no errors ~\cite{1}. In-order says the information is delivered to the receivers application layer in the same order it was sent. Without duplication means that the receiver will not receive the same data twice ~\cite{1}~\cite{2}.\\

The TCP mechanism to guarantee reliability is achieved via acknowledgements (ACKs) from the receiver to the sender. The ACKs let the sender know that data has been successfully delivered. If the sender does not receive an ACK back for data sent within a certain timeout period, TCP makes sure the data is retransmitted until an ACK is received (reliable delivery) ~\cite{1}~\cite{12}. \\

TCP also has what is known as \emph{"slow start”}, which is part of a congestion control mechanism to prevent the sender flooding the network and overwhelming the buffer queues of intermediate routers which will lead to datagrams/packets being dropped. Slow start ensures the senders initially send out only a few packets over the network, gradually increasing the number of packets sent out dependent on the acknowledgements (ACKs) returned. The rate of growth of packets sent depends on the TCP variant being used. For TCP Reno, for instance, the growth is exponential. It will send one packet, receive an ACK and then send two, receive an ACK then send four and so forth, increasing exponentially until an ACK fails to return ~\cite{1}~\cite{12}. \\

When an ACK fails to return in the allotted timeout period, TCP initiates "exponential backoff" which halves the number of datagrams being sent. TCP will continue to halve this output until all ACKs are being returned once again. The exponential backoff is then replaced with "congestion avoidance" which will increase the sender's output linearly to avoid congestion and packet loss ~\cite{1}. The linear growth of sender output at this point, rather than exponential, is TCP's way of trying not to overshoot the maximum capacity of the network that would lead to packet loss and another "exponential backoff". The goal of these TCP mechanisms is to reach a network "sweet spot", so to speak, that allows the senders to make the maximum use of the network link available as possible without packet loss ~\cite{1}~\cite{12}. With terrestrial-based communications between terrestrial senders and receivers, these TCP mechanisms work well and mainly achieve that goal. \\

TCP encounters many problems, however, when traversing via a narrowband satellite link. Firstly, the satellite link is a natural bottleneck because most terrestrial applications run on Internet speeds of 1GB to 2GB per second feeding into a satellite link that has a 16Mb per second connection speed ~\cite{4}~\cite{5}. This factor exacerbates one of the original problems TCP is designed to mitigate. That problem being the intermediate router buffers being filled up too quickly by the sender output, as we now have a situation analogous to having a water funnel designed to allow a flow of 160ml per second trying to cope with a bucket of water being poured into it at 2 Litres per second. Overflow is inevitable, and it will happen much sooner than if it were a TCP communication between senders/receivers on land ~\cite{8}~\cite{10}. \\

The bottleneck problem mentioned above leads us to the other related major issue TCP encounters with satellite links; namely the latency it must contend with (typically around 500ms round trip time). When the terrestrial sender's packet rate quickly exceeds the satellite link’s transmission rate, packets are put in a queue. Eventually, the queue may overflow, losing packets which will mean the sender does not receive the corresponding ACK, and the sender will need to retransmit the packet ~\cite{11}~\cite{13}. The added problem that the link latency causes here is that even packets that get through to the receiver may end up being retransmitted because of the ACKs timeout when they take too long to get back to the sender. Therefore, the satellite link presents an overly exaggerated account of packet loss to the sender causing issues that would not be present in terrestrial TCP environments ~\cite{13}.\\

The sender believes the packets have been lost and retransmits. These false timeouts and subsequent retransmissions trigger "exponential backoff" needlessly. (The satellite link is also prone to random link losses unrelated to congestion which also trigger "exponential backoff" unnecessarily and contribute to the overall inefficient use of the network, but this will be discussed later in the thesis) ~\cite{13}~\cite{14}. At any one time, the satellite link carries multiple simultaneous TCP connections, so all the senders exponentially back off simultaneously.  The queues are then given a chance to clear as the senders drastically reduce output, but they sit idle for longer than desired as the senders cease emitting over the link. This simultaneous backoff as a result of false timeouts caused by latency is a serious contributing factor to one of the significant issues previously mentioned with the TCP over satellite link communication; namely the underutilisation of the link ~\cite{13}.\\


At the same time, all the delayed ACKs that the sender regarded as lost (false timeouts and retransmissions) arrive back at the senders at more or less the same time. This heavy bombardment of ACKs causes the senders to ramp up again simultaneously, and via slow start, all rapidly increase their packet output over the large latency narrowband satellite bottleneck link which floods the buffer queues causing another cycle of congestion resulting in false timeouts and retransmissions and the subsequent excessive exponential backoff. The perpetuation of a TCP cycle of backing off and ramping up once again is established and fixed in place. This effect is called \emph{queue oscillation} which is quite reasonable in networks to an extent ~\cite{4}~\cite{5}. Provided the queue oscillation occurs only occasionally and relatively slowly, resulting in minimal packet loss, then it is an acceptable state to be in. The queue never overflows or empties for very long. However, due to the exaggerated account of packet loss triggered by the issues as mentioned earlier with TCP over satellite links, the result is \emph{extreme queue oscillation} which is undesirable. The satellite link switches between extreme states of idleness and link underutilisation and extreme state of network flooding senders resulting in significant packet loss ~\cite{4}~\cite{5}.\\


Ideally, we want our senders to receive ACKs for delivered data faster to eliminate false timeouts and retransmissions or cut them significantly. If the sender could be informed of delivered packets sooner, it would help ease queue overflow and reduce the extremity of the queue oscillation caused by the issues mentioned earlier. This problem has been a topic of research for decades ~\cite{16}. One of the solutions, which this thesis will revolve around, is a Performance Enhancing Proxy (PEP) that can sit between the client and the server-side of the satellite link ~\cite{6}. There are many variations of PEP's proposed over the years, but the one considered as most useful by many is the connection breaking PEP (a.k.a, TCP splitting PEP) ~\cite{14}. To the server, the PEP will seem to be the client. Likewise, to the client, the PEP will appear to be the server ~\cite{6}. The PEP will receive the packets and send an ACK to tell the sender that client has received the data (or vice versa). The PEP will then pass the data in a \emph{separate connection} to the client pretending to be the sender. This reduces the latency on each of the two connections and lessens the effects of queue oscillation, but breaks the fundamental end-to-end principle of the Internet. This breaking of the principle is problematic in itself and will be detailed further in this thesis~\cite{6}. \\

\emph{The motivation:} This research could be of value to specific regions in the Pacific where inhabitants rely on satellite link Internet (e.g., Niue, Cook Islands, Tuvalu, Kiribati, …) ~\cite{3}~\cite{4}. NZ, for instance, is investing heavily in bringing speedier broadband services via fibre optics over the next few years. Pacific islands are being left far behind the world, and this is one motivation for this project from the authors point of view. It is also an topic that can be applied to other areas where the performance of TCP protocols can suffer degradation such as satellite link connectivity to ships and planes ~\cite{4}.

\section{Research Objectives/Questions}\label{Objective}
The Research Objectives/Questions are as follows:\\

The research will contribute to the already existing field of study on Performance Enhancement Proxies (PEP). Many different types of PEP have been developed to enhance the poor performance of Internet protocols caused by specific characteristics of a link or subnetwork on the path ~\cite{6}. This thesis will focus only on satellite link PEPs and aims to develop an alternative to two existing open source solutions called TCPEP and PEPsal. These PEP solutions essentially create two distinct connections between the PEP, the server and the PEP and client and are therefore connection breaking ~\cite{6}. The solutions use conventional TCP socket protocols. This breakage of the fundamental end to end principle of the Internet is a problem ~\cite{13}~\cite{14}, and this thesis will develop a platform that avoids the connection breakage. \\

\section{Proposed Solution}\label{Solution}
This research aims to create a non-connection breaking performance-enhancing proxy. I will be using raw sockets, so I can customise the protocols to suit our view of the network. For example, we can eventually modify the ACK advertised window to be smaller if we think we sent enough packets to control the flow more effectively. We can create copies of the data packets received in a cache and send an ACK immediately, even if we have not received an ACK back from the real receiver yet. When the real ACK arrives, we can delete the copy. If the real ACK does not arrive, we can resend the data packet copy from cache.\\

We have an advantage over the other PEP solutions using conventional TCP protocols because we will be able to program flow control with knowledge that the client and server may not have. For instance, with Pacific Island scenarios with a very small local latency, so we will know the approximate Round Trip Time (RTT) from the PEP to the clients. Hence we can manipulate flow control accordingly for maximum efficiency ~\cite{4}.

\subsection{Novelty}\label{Novelty} 
Most, if not all, PEPs that are not proprietary or commercial are connection breaking PEPs (TCP splitting) ~\cite{6}. This means the PEP sits in the middle of two connections and acts as a proxy mimicking the side that client/sender thinks they are directly communicating to and this can cause problems ~\cite{6}. For instance, if one server goes down, but the client has sent data to the server and received confirmation from the PEP (via a PEP generated ACK) that the server has received the data, this client will falsely believe the data has been delivered. The solution presented in this thesis differs from these PEPs because it will not break the connection between either side ~\cite{6}. 
%------------------------------------------------

\section{Thesis Structure}\label{Structure}
This section will give a breakdown of the thesis structure; In chapter one, we have examined the motivation and problems for this research on Performance Enhancing Proxies(PEP). Research objectives have been outlined, and the proposed solution to the problem has been explained briefly. \\

Chapter 2, this thesis will provide the reader with some background to give context. The section will provide a synopsis of the TCP/IP protocol suite and will then cover TCP traffic control methods starting with a brief overview of TCP flow control and ending with the TCP congestion control/Slow Start which is the more pertinent of the two control methods for this thesis. The thesis will also go into more detail about the Internet problems in Pacific Islands. Finally, this chapter will conclude with a summary of "Other concepts used in this thesis" which will provide references to many topics relating to the thesis that the reader may choose to conduct further research on..  \\

Chapter 3 will begin by defining a performance enhancing proxy before delving into a literature review to give a history brief of the satellite link issues and types of solutions that have been tried and tested in the past, including the variety of PEPs. Next, the chapter will look at the different PEP categorisations and explain where our PEP fits. Finally, this chapter will look at past implementations of PEPs in existence now, how they are categorised in comparison to our PEP and compare and contrast. \\ 

Chapter 4 is about the methodology we use in creating and experimenting with the PEP. It will cover a specific type of socket programming (RAW SOCKET) how it differentiates from socket programming as well as how it is essential for our proposed solution. It will show the steps I have taken to create the PEP and how a non-connection breaking PEP applies to the Pacific Island network infrastructure. It will guide the reader through the steps of creating a packet sniffer, a packet injector and then through the creation of an ARP table and ARP requests packets for forwarding packets. Finally, the reader will be given a rundown of our proposed non-connection breaking PEP and taken through a step by step walkthrough of the PEP logic (i.e.the actions the PEP takes as it encounters incoming packets) followed by the implementation of that logic and reasoning for it.\\

Chapter 5 looks at our findings, will give a detailed description of the University of Auckland's Pacific Island Satellite Simulator testbed. This chapter discusses the overall project findings and it will be using the PEP on the University of Auckland Pacific Island satellite simulator and testing these results against PEPsal experiments also used on this testbed.\\

Finally, Chapter 6 concludes the thesis by giving an overall research summary and discussing its practical implications. We analyse the results, strengths and weaknesses/limitations of my research and end with suggestions for further research avenues that could be explored to improve the PEP.

